%!TEX root = ../proteoform_nomenclature.tex
%---------------------------------------------------------------------
%	Introduction
%---------------------------------------------------------------------

\newpage
\section{Introduction}

Our subcommittee was tasked to propose a nomenclature that can be used to write the sequence of a given proteoform. A proteoform is a specific set of amino acids arranged in a particular order, which may be further modified (cotranslationally, posttranslationally, or chemically) at designated locations. We met via phone, working on electronically shared documents, for an hour a week between November and December 2016. Our task is described below:
\\

\noindent\underline{Task:} provide an unambiguous notation for writing an individual proteoform. The notation must:
\begin{itemize}
\item Be Human readable. Suitable for display in written document or presentation.
\item Be Machine parsable. 
\item Contain the complete amino acid sequence of the observed proteoform
\item Specify the location and type of each modification.
\end{itemize}